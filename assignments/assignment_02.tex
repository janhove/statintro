\documentclass[10pt]{article}\usepackage[]{graphicx}\usepackage[]{xcolor}
% maxwidth is the original width if it is less than linewidth
% otherwise use linewidth (to make sure the graphics do not exceed the margin)
\makeatletter
\def\maxwidth{ %
  \ifdim\Gin@nat@width>\linewidth
    \linewidth
  \else
    \Gin@nat@width
  \fi
}
\makeatother

\definecolor{fgcolor}{rgb}{0.345, 0.345, 0.345}
\newcommand{\hlnum}[1]{\textcolor[rgb]{0.686,0.059,0.569}{#1}}%
\newcommand{\hlstr}[1]{\textcolor[rgb]{0.192,0.494,0.8}{#1}}%
\newcommand{\hlcom}[1]{\textcolor[rgb]{0.678,0.584,0.686}{\textit{#1}}}%
\newcommand{\hlopt}[1]{\textcolor[rgb]{0,0,0}{#1}}%
\newcommand{\hlstd}[1]{\textcolor[rgb]{0.345,0.345,0.345}{#1}}%
\newcommand{\hlkwa}[1]{\textcolor[rgb]{0.161,0.373,0.58}{\textbf{#1}}}%
\newcommand{\hlkwb}[1]{\textcolor[rgb]{0.69,0.353,0.396}{#1}}%
\newcommand{\hlkwc}[1]{\textcolor[rgb]{0.333,0.667,0.333}{#1}}%
\newcommand{\hlkwd}[1]{\textcolor[rgb]{0.737,0.353,0.396}{\textbf{#1}}}%
\let\hlipl\hlkwb

\usepackage{framed}
\makeatletter
\newenvironment{kframe}{%
 \def\at@end@of@kframe{}%
 \ifinner\ifhmode%
  \def\at@end@of@kframe{\end{minipage}}%
  \begin{minipage}{\columnwidth}%
 \fi\fi%
 \def\FrameCommand##1{\hskip\@totalleftmargin \hskip-\fboxsep
 \colorbox{shadecolor}{##1}\hskip-\fboxsep
     % There is no \\@totalrightmargin, so:
     \hskip-\linewidth \hskip-\@totalleftmargin \hskip\columnwidth}%
 \MakeFramed {\advance\hsize-\width
   \@totalleftmargin\z@ \linewidth\hsize
   \@setminipage}}%
 {\par\unskip\endMakeFramed%
 \at@end@of@kframe}
\makeatother

\definecolor{shadecolor}{rgb}{.97, .97, .97}
\definecolor{messagecolor}{rgb}{0, 0, 0}
\definecolor{warningcolor}{rgb}{1, 0, 1}
\definecolor{errorcolor}{rgb}{1, 0, 0}
\newenvironment{knitrout}{}{} % an empty environment to be redefined in TeX

\usepackage{alltt}

\title{Aufgabenblatt 1}
\usepackage{setspace}
\setstretch{1.3}
\usepackage{parskip}

\usepackage[utf8]{inputenc}
\usepackage[T1]{fontenc}
\usepackage{ngerman}
\usepackage{hyphenat}
\usepackage{natbib}

\usepackage[a4paper, top=3cm, bottom=3cm, left=2cm, right=3cm, marginparwidth = 4cm, marginparsep = 0.5cm]{geometry}

\usepackage[small, explicit]{titlesec}
\usepackage[normalem]{ulem}
\titleformat{\paragraph}[runin]{\normalfont\normalsize\bf}{\theparagraph}{1em}{#1}
\titleformat{\subparagraph}[runin]{\normalfont\normalsize}{\thesubparagraph}{1em}{\uline{#1}}
\titlespacing*{\subparagraph}{0pt}{1ex}{1em}

\usepackage{marginnote}
\renewcommand*{\marginfont}{\footnotesize}

\usepackage{tikz}
\usetikzlibrary{calc, intersections, arrows}

\usepackage{amsmath}
\usepackage{amsfonts}
\usepackage{amssymb}
\usepackage{amsthm}
\usepackage{mathtools}
\usepackage{mathrsfs}

\usepackage{tgpagella}
\usepackage[sc]{mathpazo}

\newtheorem*{lemma}{Lemma}
\newtheorem*{claim}{Behauptung}
\let\oldproofname=\proofname
\renewcommand*{\proofname}{\rm\bf{Beweis}}

\author{Einführung quantitative Datenanalyse}
\date{\today}

\usepackage{booktabs}
\usepackage{fancyhdr}
\usepackage{lastpage}
\pagestyle{fancy}
\fancyhf{}
\makeatletter
\let\runauthor\@author
\let\runtitle\@title
\makeatother
\lhead{\runauthor}
\rhead{S.~\thepage~von \pageref{LastPage}}
\chead{\runtitle}

\usepackage[shortlabels]{enumitem}

\usepackage{microtype}
             
\newcommand{\N}{\mathbb{N}}
\newcommand{\Z}{\mathbb{Z}}
\newcommand{\Q}{\mathbb{Q}}
\newcommand{\R}{\mathbb{R}}
\newcommand{\K}{\mathbb{K}}
\newcommand{\C}{\mathbb{C}}
\newcommand{\df}{\,\textrm{d}}
\newcommand{\pr}{\,\textrm{pr}}
\DeclareMathOperator*{\esssup}{ess\,sup}
\DeclareMathOperator*{\supp}{supp}

\newcommand*{\QED}[1][$\square$]{%
\leavevmode\unskip\penalty9999 \hbox{}\nobreak\hfill
    \quad\hbox{#1}%
}
                                
\hyphenation{}

\usepackage{hyperref}
\IfFileExists{upquote.sty}{\usepackage{upquote}}{}
\begin{document}

%%%%%%%%%%%%%%%%%%%%%%%%%%%%%%%%%%%%%%%%%%%%%%%%%%%%
\section*{Zu den Hausaufgaben}
In R Daten analysieren ist Übungssache.
Deswegen sollten Sie die R-Befehle, die im
Skript vorkommen, selber ausprobieren
(abtippen, nicht kopieren und einkleben!)
und links und rechts mal etwas ändern,
um zu schauen, was sich im Output verändert.
Weiter sollten Sie sich jede Woche einige
der Aufgaben im Skript zur Brust nehmen
und diese im Idealfall mit anderen Studierenden
diskutieren.

Ein paar der Aufgaben im Skript stelle ich
jeweils in leicht abgeänderter Form aufs
wöchentliche Aufgabenblatt. Diese Aufgaben
verstehen sich als fakultativ, bieten Ihnen
aber die Gelegenheit bei mir Feedback
zu Ihrem R-Code einzuholen. So können Sie
hoffentlich möglichst schnell anfangen,
einen guten Stil zu pflegen.
Unter einem guten Stil verstehe ich
gut strukturierten, deutlich geschriebenen
Code, mit dem die Analyse reproduziert werden kann.

In diesem ersten Aufgabenblatt liegt der Fokus
auf der Reproduzierbarkeit der `Analyse'.
Sie sollten (ausserhalb von R) einen kleinen
Datensatz anfertigen und diesen dann in R
einlesen und anzeigen. Die Schritte, die Sie
in R ausführen, können Sie dann ganz einfach
dokumentieren, indem Sie die entsprechenden Befehle
in ein Skript eintragen. Um sicher zu stellen,
dass alle Befehle funktionieren, können Sie dann 
automatisch eine HTML-Datei herstellen, die sowohl
die Befehle als auch die Ergebnisse dieser Befehle
enthält.

\section*{Daten eintragen, einlesen und anzeigen}
Slavin et al.\ (2011) berichten die Ergebnisse einer mehrjährigen Evaluationsstudie,
in der zwei Unterrichtsprogramme miteinander verglichen wurden.
Schülern und Schülerinnen (SuS) in beiden Programmen wurden unter anderem
ein spanischer und ein englischer Vokabeltest vorgelegt, und zwar
in der 1., der 2., der 3. und der 4. Klasse. Die vier Tabellen 
unten zeigen einen Teil der Ergebnisse, die Slavin et al.\ (2011) berichten; es handelt sich
dabei um die durchschnittlichen Vokabeltestergebnisse der getesteten SuS.

\begin{enumerate}
\item Tragen Sie diese Daten in ein Spread\-sheet im 
langen Format ein. Jede Zeile soll das Ergebnis in einer einzigen 
Klasse, in einer einzigen Sprache und von einem einzigen Programm 
enthalten. Sie brauchen also 16 Zeilen mit Daten und eine Zeile 
mit passenden Spaltennamen.

\item Speichern Sie dieses Spread\-sheet im CSV-Format.
Lagern Sie diese CSV-Datei in dem Unterordner
\texttt{data} in Ihrem Projektordner ab.

\item Erstellen Sie nun ein (vorläufig leeres) R-Skript und speichern 
Sie dieses in einem neuen Unterordner \texttt{assignments} in Ihrem R-Projekt.
Sie können das Skript beispielsweise \texttt{IhrName\_blatt01.R} nennen.
Tragen Sie die Befehle, die Sie für die nächsten Teile dieser
Aufgabe brauchen, in dieses neue Skript ein.

\item Lesen Sie das Spread\-sheet in R ein.

\item Kontrollieren Sie, ob das Spread\-sheet richtig eingelesen wurde.

\item Zeigen Sie in R nun nur die Englischergebnisse im \textit{transitional bilingual}-Programm an.

\item Zeigen Sie die Spanischergebnisse in der 1. und 2. Klasse im \textit{English immersion}-Programm an.

\item Führen Sie den Befehl \texttt{sessionInfo()} aus.
\end{enumerate}

Wenn das alles geklappt hat, klicken Sie \texttt{File > Compile Report\dots} und wählen Sie
das HTML-Format aus. So erzeugen Sie eine HTML-Datei, die sowohl die R-Befehle
als auch den Output dieser Befehle zeigt. Reichen Sie diese HTML-Datei auf Moodle ein.

\begin{table}[h]
\centering
\caption{Vokabeltestergebnisse 1.\ Klasse.}
\label{tab:slavin_1}
\begin{tabular}{@{}lcc@{}}
            & Transitional bilingual  & English immersion \\
 English    & 74.98                   & 79.90 \\
 Spanish    & 99.85                   & 90.19
\end{tabular}
\end{table}

\begin{table}[h]
\centering

\caption{Vokabeltestergebnisse 2.\ Klasse.}
\label{slavin_2}
\begin{tabular}{@{}lcc@{}}
            & Transitional bilingual  & English immersion \\
 English    & 80.40                        & 81.13 \\
 Spanish    & 92.94                        & 87.54
\end{tabular}

\end{table}
\begin{table}[h]
\centering
\caption{Vokabeltestergebnisse 3.\ Klasse.}
\label{slavin_3}
\begin{tabular}{@{}lcc@{}}
            & Transitional bilingual  & English immersion \\
 English    & 84.76                        &  85.45\\
 Spanish    & 92.86                        & 85.64
\end{tabular}
\end{table}

\begin{table}[h!]
\centering
\caption{Vokabeltestergebnisse 4.\ Klasse.}
\label{slavin_4}
\begin{tabular}{@{}lcc@{}}
            & Transitional bilingual  & English immersion \\
 English    & 88.07                        &  90.36\\
 Spanish    & 91.00                        & 86.27
\end{tabular}
\end{table}

\end{document}

